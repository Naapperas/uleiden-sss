\section{Results and Discussion} \label{results}

After running the tool developed by the authors to aid in programmatically generating SBOMs for the example repositories chosen, a total of 13 files have been generated: 9 CycloneDX SBOMs and 4 SPDX SBOMs. No SWID Tag files could be generated using the tool. Nonetheless, a SWID Tag file could be generated outside of the "standard" process. Further discussion is divided per SBOM Standard.

\subsection{CycloneDX} \label{results:cdx}

When comparing the outputs of the CycloneDX tools (\verb|cdxgen|, \verb|syft|, and \verb|build-info-go|), all produce SBOMs compliant with CycloneDX Version 1.6, enabling easier comparisons. During this analysis, \verb|build-info-go| stopped generating SBOMs for Apache Kafka, so an older SBOM was used.

Among the three tools, \verb|build-info-go| generates the least detailed SBOMs, containing only basic metadata and incomplete component information. For instance, it failed to identify any components or dependencies for \textit{numpy}, likely due to the limited maturity of the BuildInfo format and its conversion to CycloneDX.

\verb|syft| produces the largest SBOMs but lacks completeness in dependency information, as seen with \textit{numpy} and \textit{Kubernetes}. While some dependencies are reported for \textit{Kafka}, they appear incomplete. However, \verb|syft| excels in generating extensive component metadata, including licensing, package URLs, CPE entries, and custom properties.

\verb|cdxgen|, developed by the CycloneDX team, offers superior detail, capturing metadata about the analyzed components and the development lifecycle. It matches \verb|syft| in component metadata while adding features like component hashes, source usages, and sometimes "scope" and evidence fields. Dependency information mirrors \verb|syft|, but \verb|cdxgen| also includes an \textit{annotations} field, suggesting the potential for richer data.

All three tools remain incomplete compared to the comprehensive CycloneDX Object Model, which includes fields like release notes, Service objects, and vulnerability information.

In summary, \verb|build-info-go| is a user-friendly starting point for generating CycloneDX SBOMs. \verb|syft| is a versatile open-source tool with strong SPDX interoperability, while \verb|cdxgen| delivers the most detailed SBOMs. Combining outputs from \verb|syft| and \verb|cdxgen| could enhance SBOM quality. Despite their merits, these tools fall short of the CycloneDX standard's full potential due to the nascent nature of SBOM adoption.

A tool exists for CycloneDX SBOM comparisons but is limited to component-level analysis. Due to repeated patterns in SBOMs, manual analysis proved more effective.

\subsection{SPDX} \label{results:spdx}

\subsection{SWID} \label{results:swid}
Generating SWID tags turned out to be an unfeasible task due to the lack of proper tools and documentation. The tools provided by NIST only support Java projects built with Maven, which is too restrictive since the largest projects use Gradle nowadays, while the tools found on GitHub require implementing a custom portion of code in the project to generate the tags, which is not as straightforward as the other tools and it might be unsuitable for complex projects.
In addition to this, the documentation provided by NIST is often not very clear on how to use the tools, which makes it even harder to generate the SWID tags.

Out of all the previously mentioned tools, only \verb|swid-maven-plugin| was able to generate a tag file for \href{https://github.com/MithunTechnologiesDevOps/maven-web-application}{this} Maven project.
We also tried to use \verb|swid-generator| to generate a SBOM for Kafka, since it is written in Java using Gradle. However, unfortunately this turned out to not be possible for the reasons mentioned above.