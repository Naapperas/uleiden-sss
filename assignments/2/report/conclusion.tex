\section{Conclusion}

Out of the 3 analyzed standards, CycloneDX appears to be the most mature and most used one, which can be seen by the existing tool support. Official tools are constantly being developed and improved. Being backed by a large organization (OWASP Foundation) provides CycloneDX with the resources necessary to grow and invest in further developments. On the other end of the spectrum lies SWID, which has the most critical aspects in usage, as testified by our unsuccessful attempts to generate SWID tags and the lack of variety in tools available for this purpose. This is a clear indicator that this standard is not as widely adopted as the others, as also shown in this previous research \cite{article:software-bom} where only the 10\% of the interviewed people actually used SWID.  This is a problem that needs to be addressed, as SWID is a useful standard for software identification and its usage should be encouraged by making it easier to generate the tags.

Future work could take the methodology and developed tools of this article and apply them to a bigger variety of SBOM generation tools and/or sample repositories (performing more tests would be unfeasible due to the amount of combinations we would have to analyze). Furthermore, and more specifically for the case of SPDX, future work could use the official \verb|cyclonedx-cli| tool to convert CycloneDX SBOMs into SPDX SBOMs and conduct further analysis from there.