\section{Introduction}

Software Chain Security is an increasingly important challenge to tackle as early as possible in the Software Development process due to the catastrophic effect that vulnerabilities or security issues in downstream dependencies can cause to any software product. \needsRef It is therefore crucial from a development standpoint to be aware of what components are included in any software project and the potential vulnerabilities that they might introduce.
Many solutions have been developed to aid in that process, the prime example being \emph{package-managers} \cite{article:package-manager}, tools that help managing software dependencies and version conflicts between dependencies. Examples of such package managers are NPM (Node Package Manager) and Cargo \cite{tools:pkg-mngr:npm,tools:pkg-mngr:cargo}.

Another solution to dependency and vulnerability tracking are \emph{Software Bills of Materials} (SBOMs) \cite{article:concept:sbom-2}, detailed listings of dependencies, their relations, licensing information and other metadata pertinent to software products. These provide a standard format to work and process dependency information, allowing easier communication and shareability.
Currently, 3 mainstream SBOM standards exist: OWASP's \emph{CycloneDX}, Linux Foundation's \emph{SPDX} and NIST's \emph{SWID} \cite{standards:sbom:cyclonedx,standards:sbom:spdx,standards:sbom:swid}.

In previous work \cite{article:concept:sbom-sota-review}, the authors performed a critical comparison on 5 articles from the literature \cite{article:business-sbom,article:software-bom,article:sbom-required,article:sbom-study,article:sboms-issues-solutions} based on author-defined metrics. The authors provide insights on the state-of-the-art regarding SBOMs and provide guidelines for SBOM adoption and development, as well as future work that can be done in further research. \red{Should add more sentences/explain this further}

In an attempt to complement \cite{article:concept:sbom-sota-review}, this article reports on a hands-on comparison of the three mainstream SBOM standards by making use of available tools for each standard. SBOM generation tools for each standard were used on a set of N major Open-Source repositories found on GitHub and the resulting SBOM output files were compared, both between standards as well as between the tools of each standard.