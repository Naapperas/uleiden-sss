\section{Methodology}

In this section, we detail the methodology used to perform the hands-on comparison between SBOM standards. In \ref{methodology:standards-tools} we discuss the different available standards and the tools chosen to generate SBOMs for each one of them. Based on the tools picked we chose, in \ref{methodology:repositories}, the repositories on which the SBOMs will be generated.

\subsection{Standards and tools} \label{methodology:standards-tools}

Each one of the three standard formats focuses on a specific part of the software supply chain, which can be reflected in the (meta)data that each standard stores and processes. The tools developed for each each standard also reflect these decisions. \red{I felt that I had to fill this in with something}

\subsubsection{CycloneDX}

CycloneDX \cite{standards:sbom:cyclonedx} is an SBOM standard format developed by the CycloneDX Core Working Group and backed by the OWASP Foundation with a focus on "cyber-risk reduction" \cite{standards:sbom:cyclonedx} and security \cite{article:sbom-study}. The standard supports writing BOMs for several domains of software development, such as Software BOMs (SBOMs), Cryptographic BOMs (CBOMs), Software-as-a-Service BOMs (SaaSBOMs), among others. Over 200 tools related to CycloneDX's SBOM format are available at \href{https://cyclonedx.org/tool-center/}{\underline{CycloneDX's official tool webpage}}. \red{Should this be a reference instead?}

For this hands-on comparison, we limited our search to \emph{Open-Source} tools as these are free to access and use. Out of 172 listed Open-Source tools, 2 were chosen: CycloneDX \verb|cdxgen|\cite{repository:cyclonedx:cdxgen} and \verb|build-info-go| \cite{repository:cyclonedx:build-info-go}.

Other tools exists but they are either unrelated (SBOM analysis, VEX generation, \dots), too specific (official SBOM generators for several existing programming languages and build tools) or too limited on the supported development environments.

\paragraph{cdxgen} is an official tool developed and released on GitHub by the CycloneDX team, built around the idea of being a "polyglot SBOM generator that is user friendly, precise and comprehensive"
\red{Add more text?}

\paragraph{build-info-go} is a CLI tool to generate BuildInfo metadata, a custom format designed to encapsulate software components, their versions and their dependencies. The tool supports multiple languages and package managers and has the option to export the produced BuildInfo output into a valid CycloneDX JSON file.

\subsubsection{SPDX}

\subsubsection{SWID}
Software Identification (SWID) Tags \cite{standards:sbom:swid} are a standard format for identifying software components and metadatas, which can be used to generate SBOMs. Nowadays, the current standard of it is ISO/IEC 19770-2:2015 \cite{standards:swid:iso19770-2:2015} and is maintained by the ISO/IEC JTC 1/SC 7/WG 21 committee \cite{standards:swid:committee}.
A generated SWID Tag document consists of a well-organized collection of data fields that specify the software product, its version, identify the organizations and people involved in its creation and distribution, list the components that make up the software, define relationships between different software products and include additional metadata for further description.
SWID tagging differs from CycloneDX and SPDX in that it is not a full-fledged SBOM format, as it doesn't aggregate information of all softwares so it's rather a standard for identifying software components and their metadata.
\\ The recommended official NIST tools for generating SWID Tags only support Java programs built with Maven, which is too restrictive, and most of them aren't properly documented or don't fit our use case. 
Out of all the official tools available, we found the following one to be the most suitable and usable:

\paragraph{swid-maven-plugin} is a Maven plugin published by NIST \cite{repository:swid-maven-plugin} that generates SWID Tags for Java projects and is compliant with the above mentioned ISO standard. It needs to be configured in the project's \verb|pom.xml| file and additionally it requires the assembly descriptor in \verb|src/assembly/bin.xml| to be configured aswell.



\subsection{Repositories} \label{methodology:repositories}

To ensure a fair comparison between standards, we chose a representative set of major Open-Source repositories that could be analyzed by most, if not all, of the tools selected.

As such, we have picked N repositories from GitHub:

\begin{itemize}
    \item Apache Kafka \cite{repository:dataset:kafka}
    \item N/A
\end{itemize}