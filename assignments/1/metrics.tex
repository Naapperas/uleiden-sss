\section{Metrics defined} \label{metrics}

In order to critically compare the reviewed papers, some metrics need to be used that can objectively rank these papers with respect to the metrics themselves.

\noindent Since SBOM is a relatively novel concept, comparing literature on this topic should focus on finding out and exposing the most common challenges pertaining to SBOM widespread adoption and how those can be mitigated. Other interesting aspects relevant for SBOM adoption is how generalized these problems are, since problems that are too specific to a certain domain only provide value for that domain and cannot be reused in other contexts.

\noindent As such, the metrics we have devised for our critical comparison of the papers we reviewed are:

\paragraph{Metrics of Quality}\mbox{}\\
Each of the following metrics will be rated on a scale from 1 to 5, where 1 means the metric is not met at all, and 5 means the metric is met perfectly.
\begin{enumerate}
    \item \textbf{Aims}: are the aims/research questions of the paper clearly defined?
    \item \textbf{Conclusions}: are the conclusions drawn from the study findings valid and do they align with the aims of the paper?
\end{enumerate}

\paragraph{Metrics of Quantity}
\begin{enumerate}
    \item \textbf{Number of SBOM adoption challenges}: the number of challenges an organization or system integrator needs to face before extracting value out of the use of SBOMs directly correlates to how eager they might be to adopting SBOMs in their processes: if the benefits don't outweigh the challenges, it is not valuable, and thus not desirable, to put in the effort required to correctly adopt and practice SBOM development.
    \item \textbf{Frequency of challenges}: the number of times common challenges appear across different research papers, either quantitatively or qualitatively.
    \item \textbf{Number of solutions to standardize SBOM use}: the number of solutions proposed by the paper to standardize SBOM use is a good indicator of how much thought the authors have put into the topic and how much they have researched the topic.
\end{enumerate}

\paragraph{Metrics of methodological soundness}
\begin{enumerate}
    \item \textbf{Sampling}: is the sampling of participant clearly defined and does it provide a representative sample of the population the paper is studying?
    \item \textbf{Analysis of study findings}: are the analytic methods clear, systematic and reproducible?
\end{enumerate}


\noindent Although we believe these metrics to be valid comparison points between the papers discussed, there are some issues which could hinder the validity of the comparisons made with them, which we will discuss in section [INSERT SECTION HERE].