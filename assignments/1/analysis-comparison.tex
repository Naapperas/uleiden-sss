\section{Analysis and Comparison of reviewed papers} \label{analysis}

\noindent With a set of relevant metrics defined for the topic of SBOMs, we analyzed and compared the papers based on these criteria. Our findings are summarized in Table 1 A value of \emph{N/A} means that the relevant metric is not applicable to the paper in question or that there isn't enough data/evidence that can sustain a scoring on that metric.

\paragraph{Zahan et al. \cite{article:sbom-required}}

\begin{enumerate}
    \item
    \item
    \item
    \item
    \item
    \item
    \item
    \item
\end{enumerate}

\paragraph{Stalnaker et al. \cite{article:software-bom}}

\begin{enumerate}
    \item This study's aims are clearly defined through well-articulated research questions, focusing on identifying the current challenges practitioners face in SBOM adoption and usage, as well as exploring solutions to enhance the practicality and reliability of SBOMs. (5/5)
    \item The authors provide a clear link between the research questions and the conclusions, making the study's findings valid. However, the range of insights tailored to specific stakeholders is occasionally generalized, resulting in some conclusions being somewhat superficial. (4/5)
    \item The paper identifies the following challenges: (12)
            \begin{enumerate}[label=\Alph*.]
                \item Complexity of SBOM specifications;
                \item Determining data fields to include in SBOMs;
                \item Incompatibility between SBOM standards;
                \item Keeping SBOMs up to date;
                \item Insufficient SBOM tooling;
                \item Inaccurate and incomplete SBOMs;
                \item Verifying SBOM accuracy and completeness;
                \item Differences across ecosystems and communities;
                \item SBOM completeness and data privacy trade-off;
                \item SBOMs for legacy packages and repositories;
                \item Inability to locate dependencies for SBOMs;
                \item Unclear SBOM direction and low adoption.
            \end{enumerate}
    \item %to do
    \item The researchers identified the following solutions: (4)
            \begin{enumerate}[label=\Alph*.]
                \item Multi-dimensional SBOM specifications;
                \item Enhanced SBOM tooling and build system support;
                \item Strategies for SBOM verification;
                \item Increasing incentives for SBOM adoption.
            \end{enumerate}
            However, these don't account for challenges A, J and K, as they require additional research to be addressed effectively, as mentioned in the paper.
    \item The sampling of participants is well explained and included 5 groups: SBOM Community and Adopters, Developers of Critical Open Source Systems, CPS Developers and Researchers, AI/ML Developers and Researchers, Legal Practitioners.
          They sent 4.4k surveys and received 229 responses, which after filtering resulted in 150 valid ones. The authors also conducted 8 interviews with respondents who gave detailed replies highlighting interesting use cases, challenges, and potential solutions; demonstrated experience in their field; diversified the interviewee pool in terms of their role.
          The sampling method is really good, but the number of participants is not enough to generalize the results for the whole industry. (4/5)
    \item The researchers conducted their studies and designed the questionnaires considering previous literature on SBOMs and guidelines for survey design. Hence, the methodology is well-founded and reproducible. (5/5)
    \item
\end{enumerate}

\paragraph{Xia et al. \cite{article:sbom-study}}

\begin{enumerate}
    \item The aims of this study are clearly defined when defining the research questions, and the potential (and effective) contributions of the study reflect this. (5/5)
    \item Since the survey data was compared against interview results and conclusions were drawn from analyzing both, we are confident in their validity. The link between research questions and conclusions is logically sound, so this paper scores a 5 on this metric. (5/5)
    \item Lack of standard format extensibility, potential for attackers to use SBOMs as an attack "guide" or lack of SBOM education by industry professionals are identified as the 3 main concerns analyzed. (3)
    \item %% TODO: catalog challenges to calculate frequencies
    \item The defined 3-goal model outlines steps that the authors believe are crucial to practice if SBOMs are to see more widespread adoption. (3)
    \item The study included 82 participants from diverse backgrounds and countries. Most expertise categories have over 10 participants, with some exceptions, such as Researchers. However, each Researcher has at least 10 years of experience in software engineering, making their assessments highly representative (4/5).
    \item The data analysis methods used for both the interviews' and online surveys' responses are backed by literature articles, so they are easily reproducible. (5/5)
    \item %% TODO: uhhh
\end{enumerate}

\paragraph{Bi et al. \cite{article:sboms-issues-solutions}}

\begin{enumerate}
    \item The aims/research questions of this study are extensively discussed and clarified (5/5)
    \item Since conclusions are drawn from the authors' analysis of the collected data and they help answer the defined research questions we can confidently say they are valid and thus this paper scores a 5 on this metric. (5/5).
    \item The paper identifies three categories of SBOM issues, with the second and third categories further divided into subcategories, resulting in a total of eight distinct SBOM issues. Notably, issues are reported as percentages, while solutions are presented as measurable counts. Despite these differing metrics, we believe the reviewed papers can still be compared effectively as is. (8)
    \item
    \item The authors explicitly state that they found "33 high-level solutions for the SBOM-relevant issues and their main design problem". It can be argued whether these are actual solutions or steps towards solutions but we will respect the author's own assessment. (33)
    \item Even though the data collection method is clearly outlined, it comes from repository mining, not participant interactions. (N/A)
    \item The data collection process is thoroughly explained and the analysis method is backed by literature references based on \emph{Grounded Theory}. (5/5)
    \item
\end{enumerate}

\paragraph{Kloeg et al. \cite{article:business-sbom}}

\begin{enumerate}
    \item
    \item
    \item
    \item
    \item
    \item
    \item
    \item
\end{enumerate}

%% TODO: fill this in
\begin{table}[h]
    \centering
    \begin{tabularx}{\textwidth}{c *{8}{|Y}}
        \multirow{2}{*}{\textbf{Paper}}                 & \multicolumn{2}{c|}{\textbf{Quality}} & \multicolumn{3}{c|}{\textbf{Quantity}} & \multicolumn{2}{c|}{\textbf{Methodology}} & \textbf{Transf.}                                                     \\
        \cline{2-9}                                     & \textbf{1}                            & \textbf{2}                             & \textbf{3}                                & \textbf{4}       & \textbf{5} & \textbf{6} & \textbf{7} & \textbf{8} \\
        \hline
        \hline
        Zahan et al. \cite{article:sbom-required}       & --                                    & --                                     & 5                                         & --               & --         & --         & --         & --         \\
        \hline
        Stalnaker et al. \cite{article:software-bom}    & 5                                    & 4                                     & 12                                        & --               & 4         & 4         & 5         & --         \\
        \hline
        Xia et al. \cite{article:sbom-study}            & 5                                     & 5                                      & 3                                         & --               & 3          & 4          & 5          & --         \\
        \hline
        Bi et al. \cite{article:sboms-issues-solutions} & 5                                     & 5                                      & 8                                         & --               & 33         & N/A        & 5          & --         \\
        \hline
        Kloeg et al. \cite{article:business-sbom}       & --                                    & --                                     & --                                        & --               & --         & --         & --         & --
    \end{tabularx}
    \caption{Comparison of reviewed papers}
    \label{tab:comparison}
\end{table}
