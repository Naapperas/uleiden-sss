\section{Overview of reviewed papers}

To assess the state-of-the-art with respect to SBOMs and their industry-wide use, we conducted a critical review of some examples from the literature, which are enumerated and described below:

\paragraph{Software Bills of Materials Are Required. Are We There Yet?} \cite{article:sbom-required} In this paper, written by Zahan et al. and published in the \emph{IEEE Security \& Privacy (Volume: 21, Issue: 2, March-April 2023)}, the authors conducted a Grey Literature (GL) review of 200 internet articles, 100 for "challenges to adopt SBOM” and 100 for "benefits to adopt SBOM” in order to assess the biggest upsides and downsides experienced in practice regarding SBOM adoption. Grey literature was chosen as the majority of content regarding SBOMs in practice cannot be found in literature but rather in online articles and blog posts. The authors came up with 5 reported benefits from SBOM adoption and 5 challenges preventing SBOM adoption.

\paragraph{BOMs Away! Inside the Minds of Stakeholders: A Comprehensive Study of Bills of Materials for Software Systems} \cite{article:software-bom} The paper, written by Stalnaker et al. and published in \emph{ICSE '24: Proceedings of the 46th IEEE/ACM International Conference on Software Engineering}, investigates the current state of Software Bills of Materials (SBOMs), which are recognized as vital tools especially after important incidents like the two mentioned ones: SolarWinds breach and Log4J vulnerabilities. The study identifies 12 major challenges related to SBOM creation and use, such as: insufficient tool support, SBOM maintenance difficulties and standard incompatibilities across different industries (as highlighted by the 138 interviews with stakeholders). The study identifies key SBOM standards (SPDX, CycloneDX, SWID) and emphasizes the need for better tools to facilitate SBOM creation, verification, and maintenance. Furthermore, it proposes 4 actionable solutions to overcome these critical problems and outlines future research directions aimed at maintaining SBOM accuracy over time and dealing with legacy systems.

\paragraph{An Empirical Study on Software Bill of Materials: Where We Stand and the Road Ahead} \cite{article:sbom-study} In this paper, written by Xia et al. and published in the \emph{ICSE '23: Proceedings of the 45th International Conference on Software Engineering}, the authors aimed to understand the state of SBOM usage and adoption in practice, conducting 17 interviews and performing a survey based on the interviews. The authors gathered information about current SBOM practicioners and what these might feel is lacking in the industry regarding SBOM practices. The study is recent, provides a systematic methodology and provides perspectives from the Software Engineering standpoint.

\paragraph{On the Way to SBOMs: Investigating Design Issues and Solutions in Practice} \cite{article:sboms-issues-solutions} This paper, written by Bi et al. and published in the \emph{ACM Transactions on Software Engineering and Methodology, Volume 33, Issue 6}, explores current practical uses and concerns/problems of SBOM in "the wild". The authors gathered data by mining several GitHub repositories and, out of those, discussions pertaining to the topic of SBOM. It was found that, generally, there are 4 phases to the SBOM lifecycle: planning, development, publication and maintenance.

\paragraph{Charting the Path to SBOM Adoption: A Business Stakeholder-Centric Approach} \cite{article:business-sbom} The paper, written by Kloeg et al. and published in the \emph{Proceedings of the 19th ACM Asia Conference on Computer and Communications Security}, analyzes the slow adoption of SBOM in improving transparency and security within software supply chains. The research identifies four key stakeholder groups—system integrators, software vendors, B2B customers, and individual developers, and also examines how their roles affect SBOM adoption. Through interviews with 16 representatives from these groups, the authors analyze the incentives, concerns, and barriers related to SBOM. System integrators and software vendors are more likely to adopt SBOMs, driven by compliance requirements and the potential to improve their reputation and the quality of supplied software. On the other hand, B2B customers and individual developers show less interest because they struggle to see its immediate value and face challenges in resources and complexity. The study reveals that the main obstacles to adoption include a lack of expertise, concerns over the time and effort required to maintain SBOMs, vulnerability misclassification, and financial costs. The paper recommends targeted regulatory interventions and improvements in SBOM tools to align incentives across stakeholders. The research concludes that while SBOMs have significant potential to enhance software security, widespread adoption will require external pressures and better tools to support stakeholders.

%% TODO: add/remove papers