\section{Introduction}

Software systems, now more than ever, outsource parts of application logic to first- or third-party \emph{dependencies} --- pieces of code that are used in conjunction or as a part of the application's business logic but that are not part of the application itself.

Keeping track of dependencies can be an arduous task, which can be more easily managed using software tools and components such as \emph{package managers} that keep track of an application's dependencies and their versions \cite{article:package-manager}, like \emph{npm} or \emph{cargo} \cite{tools:pkg-mngr:npm,tools:pkg-mngr:cargo}.

SBOMs (Software Bills of Materials) \cite{article:concept:sbom-2}, proposed by the U.S. National Telecommunications and Information Administration (NTIA), are a formal way of describing the software dependencies of an application and the relations between these dependencies. Other metadata can be attached to SBOM entries for a more comprehensive outline of the software components being dependended upon.

Currently, there are 3 SBOM standards used in practice: OWASP CycloneDX \cite{standards:sbom:cyclonedx}, Software Product Data eXchange (SPDX) \cite{standards:sbom:spdx} and Software Identification Tagging (SWID) \cite{standards:sbom:swid}. However, little consensus exists between these 3 standards, making for one of the many challenges against general SBOM adoption \cite{article:sbom-study}.

%% TODO: expand this somehow